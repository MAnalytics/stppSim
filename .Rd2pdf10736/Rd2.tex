\documentclass[a4paper]{book}
\usepackage[times,inconsolata,hyper]{Rd}
\usepackage{makeidx}
\usepackage[utf8]{inputenc} % @SET ENCODING@
% \usepackage{graphicx} % @USE GRAPHICX@
\makeindex{}
\begin{document}
\chapter*{}
\begin{center}
{\textbf{\huge Package `stppSim'}}
\par\bigskip{\large \today}
\end{center}
\begin{description}
\raggedright{}
\inputencoding{utf8}
\item[Type]\AsIs{Package}
\item[Title]\AsIs{Spatial and Temporal point patterns Simulation for Social Science Research}
\item[Version]\AsIs{0.1.0}
\item[Author]\AsIs{Monsuru Adepeju [cre, aut]}
\item[Maintainer]\AsIs{MOnsuru Adepeju }\email{monsuur2010@yahoo.com}\AsIs{}
\item[Description]\AsIs{Developed for simulating geographical point patterns within a specified spatial
and temporal configurations. The `stppSim` can be applied in a wide range of
social science domains, such as criminology and epidemiology, in which
there is a lack of access to real-life data.}
\item[Language]\AsIs{en-US}
\item[License]\AsIs{GPL-3}
\item[URL]\AsIs{}\url{https://github.com/MAnalytics/stppSim}\AsIs{}
\item[BugReports]\AsIs{}\url{https://github.com/Manalytics/stppSim/issues/new/choose}\AsIs{}
\item[Depends]\AsIs{R (>= 4.0.0)}
\item[Encoding]\AsIs{UTF-8}
\item[LazyData]\AsIs{true}
\item[Roxygen]\AsIs{list(markdown = TRUE)}
\item[Imports]\AsIs{splancs,
dplyr,
magrittr,
utils,
sf,
rgdal,
sp}
\item[RoxygenNote]\AsIs{7.1.1}
\item[Suggests]\AsIs{}
\end{description}
\Rdcontents{\R{} topics documented:}
\inputencoding{utf8}
\HeaderA{camden\_boundary}{A boundary shapefile}{camden.Rul.boundary}
\keyword{datasets}{camden\_boundary}
%
\begin{Description}\relax
A boundary shapefile
\end{Description}
%
\begin{Usage}
\begin{verbatim}
camden_boundary
\end{verbatim}
\end{Usage}
%
\begin{Format}
A boundary file (ESRI format)
\begin{itemize}

\item{} x: x coordinate
\item{} y: y coordinate

\end{itemize}

\end{Format}
\inputencoding{utf8}
\HeaderA{constrained\_spo}{Simulate spatial point origins constrained by the social configuration of the urban space.}{constrained.Rul.spo}
%
\begin{Description}\relax
Simulate event origins (EOs) on a land use map
(contrained space) with binary classes \code{1} and \code{0}, representing
active and non-active origins. An \code{active} origin can
generate events while a \code{non-active} origin can not generate
events. Each active origin is assigned
a probability value (representing the intensity) at which
the origin generates events in accordance with a specified
Pareto ratio.
\end{Description}
%
\begin{Usage}
\begin{verbatim}
constrained_spo(bpoly, p_ratio = 5,
show.plot=FALSE)
\end{verbatim}
\end{Usage}
%
\begin{Arguments}
\begin{ldescription}
\item[\code{bpoly}] (a spatialPolygonDataFrames) with a binary attribute
field \code{class}, which includes values \code{1} and \code{0}, representing
the active and non-active origins.

\item[\code{p\_ratio}] (an integer) The smaller of the
two terms of a Pareto ratio. For example, for a \code{20:80}
ratio, \code{p\_ratio} will be \code{20}. Default value is
\code{30}. Valid inputs are \code{10}, \code{20},
\code{30}, \code{40}, and \code{50}. A \code{30:70}, represents
30\% dominant and 70\% non-dominant origins.

\item[\code{show.plot}] (TRUE or FALSE) To display plot showing
base map (i.e. social configuration of the landscape, in
terms of active and non-active spaces), and the event origins.
\end{ldescription}
\end{Arguments}
%
\begin{Details}\relax
Note: The \code{class} field of 'bpoly'
will be utilized for mapping the basemap.
\end{Details}
%
\begin{Value}
Returns the event origins constraint by the
social configuration of the space
\end{Value}
%
\begin{References}\relax
\#https://online.stat.psu.edu/stat510/lesson/6/6.1
\end{References}
\inputencoding{utf8}
\HeaderA{extract\_coords}{Extracting coordinates of a polygon boundary}{extract.Rul.coords}
%
\begin{Description}\relax
Given a polygon object, the goal is to extract
the coordinates of the edges of the boundary.
\end{Description}
%
\begin{Usage}
\begin{verbatim}
extract_coords(poly)
\end{verbatim}
\end{Usage}
%
\begin{Arguments}
\begin{ldescription}
\item[\code{poly}] (a spatialPolygons, spatialPolygonDataFrames).
The polygon object must be in a
Cartesian coordinate system.
\end{ldescription}
\end{Arguments}
%
\begin{Value}
Returns the global temporal pattern
\end{Value}
%
\begin{References}\relax
https://www.google.co.uk/
\end{References}
\inputencoding{utf8}
\HeaderA{gtp}{Modeling of the Global Temporal Pattern}{gtp}
%
\begin{Description}\relax
Models the global temporal pattern (of
the point process) as consisting of the global linear
trend and the seasonality.
\end{Description}
%
\begin{Usage}
\begin{verbatim}
gtp(start_date = "01-01", trend = "stable",
slope = "NULL", first_s_peak=90, scale = 1, show.plot =FALSE)
\end{verbatim}
\end{Usage}
%
\begin{Arguments}
\begin{ldescription}
\item[\code{start\_date}] The start date of the study period.
Default value is \code{"01-01"} (i.e. January 1st). By default
the end date of the study period is set as \code{"12-31"} (i.e.
31st December). A user can specify any start date in the
format \code{"mm/dd"}. The end date is the next 365th day
from the specified start date.

\item[\code{trend}] (a character) Specifying the direction of
the global (linear) trend of the point process. Three options
available are \code{"decreasing"}, \code{"stable"},
and \code{"increasing"} trends. Default: \code{"stable"}.

\item[\code{slope}] (a character) Slope angle for an
"increasing" or "decreasing" trend. Two options
are available: \code{"gentle"} and \code{"steep"}.
Default value is \code{"NULL"} for the default trend
(i.e. \code{stable}).

\item[\code{first\_s\_peak}] Number of days before the first seasonal
peak. Default: \code{90}. This implies a seasonal cycle
of 180 days.

\item[\code{scale}] (an integer) For scaling point counts. Default: \code{1}

\item[\code{show.plot}] (TRUE or False) To show the time series
plot. Default is \code{FALSE}.
\end{ldescription}
\end{Arguments}
%
\begin{Value}
Returns the global temporal pattern
\end{Value}
%
\begin{References}\relax
\#https://online.stat.psu.edu/stat510/lesson/6/6.1
\end{References}
\inputencoding{utf8}
\HeaderA{make\_grids}{Make Square Grids System}{make.Rul.grids}
%
\begin{Description}\relax
Generates a system of square grids over a specified
spatial boundary.
\end{Description}
%
\begin{Usage}
\begin{verbatim}
make_grids(poly, size = 200,
show.output = FALSE, dir=NULL)
\end{verbatim}
\end{Usage}
%
\begin{Arguments}
\begin{ldescription}
\item[\code{poly}] (as \code{spatialPolygons}, \code{spatialPolygonDataFrames}, or
\AsIs{simple features}). A spatial polygon over
which the spatial grid is to be overlaid. Needs to be in a
cartesian CRS.

\item[\code{size}] Square grid size to be generated.
To be in the same unit associated with the \code{poly} (e.g.
metres, feets, etc.). Default: \code{200}.

\item[\code{show.output}] (logical) To show the output.
Default: \code{FALSE}

\item[\code{dir}] (character) Specifies the directory to
export the output. Default is \code{NULL}, indicating the
current working directory (cwd). A user can specify a different
directory in the format: "C:/.../folder".
\end{ldescription}
\end{Arguments}
%
\begin{Value}
Returns a spatial square grid system
in a shapefile format
\end{Value}
%
\begin{References}\relax
https://www.google.co.uk/
\end{References}
\inputencoding{utf8}
\HeaderA{poly}{Boundary Coordinates of Camden Borough of London}{poly}
\keyword{datasets}{poly}
%
\begin{Description}\relax
Boundary Coordinates of Camden Borough of London
\end{Description}
%
\begin{Usage}
\begin{verbatim}
poly
\end{verbatim}
\end{Usage}
%
\begin{Format}
A dataframe containing one variable:
\begin{itemize}

\item{} x: x coordinate
\item{} y: y coordinate

\end{itemize}

\end{Format}
\inputencoding{utf8}
\HeaderA{p\_prob}{Pareto Probability distribution}{p.Rul.prob}
%
\begin{Description}\relax
Given a specified number of points \code{n},
this function generates an \code{n} probability values
in accordance with a specified Pareto ratio.
\end{Description}
%
\begin{Usage}
\begin{verbatim}
p_prob(npoints,  p_ratio = 30)
\end{verbatim}
\end{Usage}
%
\begin{Arguments}
\begin{ldescription}
\item[\code{npoints}] (an integer) Number of points. Default is
\code{50}.

\item[\code{p\_ratio}] (an integer) The smaller of the
two terms of a Pareto ratio. For instance, for a
\code{20:80} ratio, \code{p\_ratio} will be \code{20}. Default value is
\code{20}. Input values must be \code{5}, \code{10}, \code{20},
\code{30}, or \code{40}. The 'p\_ratio'
determines the proportion of points that are the most
dominant event generators.
\end{ldescription}
\end{Arguments}
%
\begin{Value}
Returns the global temporal pattern
\end{Value}
%
\begin{References}\relax
https://www.google.co.uk/
\end{References}
\inputencoding{utf8}
\HeaderA{random\_spo}{Simulate random origins for spatial points}{random.Rul.spo}
%
\begin{Description}\relax
Simulate point origins for generating the
spatial point across the area. Each origin is assigned
a probability value (representing the relative intensity) at which
the origin generates events in accordance with a specified
Pareto ratio.
\end{Description}
%
\begin{Usage}
\begin{verbatim}
random_spo(poly, npoints, p_ratio, show.plot=FALSE)
\end{verbatim}
\end{Usage}
%
\begin{Arguments}
\begin{ldescription}
\item[\code{poly}] (a list or dataframe) A list of spatial boundary
coordinates within which the events are confined.

\item[\code{npoints}] (an integer) Number of origins (points) to simulate

\item[\code{p\_ratio}] (an integer) The smaller of the
two terms of a Pareto ratio. For example, for a \code{20:80}
ratio, \code{p\_ratio} will be \code{20}. Default value is
\code{30}. Valid inputs are \code{10}, \code{20},
\code{30}, \code{40}, and \code{50}. A \code{30:70}, represents
30\% dominant and 70\% non-dominant origins.

\item[\code{show.plot}] (TRUE or FALSE) To display plot showing
points (origins).
\end{ldescription}
\end{Arguments}
%
\begin{Value}
Returns random event origins
\end{Value}
%
\begin{References}\relax
\#https://online.stat.psu.edu/stat510/lesson/6/6.1
\end{References}
\inputencoding{utf8}
\HeaderA{regular\_poly}{A rectangular boundary coordinates}{regular.Rul.poly}
\keyword{datasets}{regular\_poly}
%
\begin{Description}\relax
A rectangular boundary coordinates
\end{Description}
%
\begin{Usage}
\begin{verbatim}
regular_poly
\end{verbatim}
\end{Usage}
%
\begin{Format}
A dataframe containing one variable:
\begin{itemize}

\item{} x: x coordinate
\item{} y: y coordinate

\end{itemize}

\end{Format}
\inputencoding{utf8}
\HeaderA{San\_Francisco}{A land use shapefile of a portion of San Francisco City, United States}{San.Rul.Francisco}
\keyword{datasets}{San\_Francisco}
%
\begin{Description}\relax
A land use shapefile of a portion of
San Francisco City, United States
\end{Description}
%
\begin{Usage}
\begin{verbatim}
San_Francisco
\end{verbatim}
\end{Usage}
%
\begin{Format}
A boundary file (ESRI format)
\begin{itemize}

\item{} landuse\_1: land use categories denoting
the social configuration of the urban space
\item{} class: a binary field indicating origins that
have the ability to generate events ('1') and origins
that lack the ability to generate points ('0').

\end{itemize}

\end{Format}
\printindex{}
\end{document}
